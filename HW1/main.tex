
%hi
\documentclass{article}
\usepackage[margin=1in]{geometry} 
\usepackage{amsmath,amsthm,amssymb,amsfonts, fancyhdr, color, comment, graphicx, environ}
\usepackage{xcolor}
\usepackage{mdframed}
\usepackage[shortlabels]{enumitem}
\usepackage{indentfirst}
\usepackage{hyperref}
\usepackage{float}
\renewcommand{\footrulewidth}{0.8pt}
\hypersetup{
    colorlinks=true,
    linkcolor=blue,
    filecolor=magenta,      
    urlcolor=blue,
}


\pagestyle{fancy}



\newenvironment{problem}[2][Problem]
    { \begin{mdframed}[backgroundcolor=gray!20] \textbf{#1 #2} \\}
    {  \end{mdframed}}


\newenvironment{solution}{\textbf{Solution}}


\lhead{Kasra Amani}
\rhead{Embedded Systems} 
\chead{\textbf{Assignment 1}}
\lfoot{Mohsen Ansari}
\rfoot{Sharif University of Technology}
\def\thesection{\alph{section}}

\begin{document}
\title{\Large Embedded Systems  \\[0.5cm]
        \bf\Large Assignment 1}
\author{\large Author: Kasra Amani\\ \normalsize Student No. 98101171 \\ \ \\}
\date{\large Date Last Edited: \today}

\makeatletter
    \begin{titlepage}
        \begin{center}
	   { \includegraphics[width=13cm]{sharif.png}}
	   {\ \\ \ \\}
        \vbox{}\vspace{5cm}
            {\@title }\\[3cm] 
            {\@author}
            {\large Instructor: \bf Prof. Ansari\\ \ \\}
            {\@date\\}

        \end{center}
    \end{titlepage}
\makeatother%not necessary but looks fancy
    \begin{problem}{1}
    	In Computation Theory, a Mealy Machine is defined as a machine which outputs a value based on 
    	both its current state and current input; in contrast, a Moore Machine's output depends only on
    	its current state.
    	Typically, a Mealy machine requires less states than a Moore machine, but more hardware is 
    	needed for a Mealy machine. In a Moore machine, the output is produced one clock cycle later
    	which is not the case in a Mealy machine. Moore machines are easier to design as well.
    \end{problem}


\end{document}